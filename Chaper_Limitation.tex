\subsection{Finding potential coarse-graining functions}
To empirically verify the current theory, it is important to know what coarse-graining functions which maps microscopic variables to macroscopic variables in the neural system. So far, in this article, we have not speculated potential candidates of the coarse-graining function (or functions) that creates NTIC in the neural system. 

\cite{Gamez2016} has systematically describe a similar issue in terms of finding data correlates of consciousness amount different 
level of abstraction. 

    \begin{ants}
        @ Our theory does not intent to solve the Hard problems. 
        @ Inverted spectrum Inverted Qualia \citep{Shoemaker1982-SHOTIS, Block1990-BLOIE, Locke1979-LOCTCE-2}
        
        @ Considering that a system has only two possible physical states $X$ and $Y$ mapping to three different qualia (we can label them as $"red"$ and $"blue"$ respectively). A system cannot rest on two states at any moment, therefore, knowing the system state at $X$ excludes the the other possible state $Y$ and receives 1 bit information. Similarly, if conscious percept is $"red"$, it rules out the $"blue"$. Therefore, information provides by the conscious percept is also 1 bit. 
        
        @ However, our theory cannot tell why $X$ and $Y$ map to $red$ and $blue$ ,respectively, rather than the other way around. 
        
        % about the coarse-graining function (aggregation function) 
        @ \cite{PFANTE.2014}, the authors discuss the the implications of the types of aggregation function (deterministic or probabilistic) and its relationship with macroscopic processes. If the macroscopic process: \rewrite{Furthermore, we could prove that observational commutativity forces the aggregation to be deterministic if the upper process is invertible, i.e., ψ is invertible, and in the case of a soft (i.e., probabilistic) aggregation the third idea of commutativity really differs from the one of observational commutativity. } 
        
        
        
        @ Our theory does not explain the mapping
    \end{ants}