\documentclass[utf8]{article}




%% Language and font encodings
\usepackage[english]{babel}
%\usepackage[utf8]{inputenc}
\usepackage[T1]{fontenc}
\usepackage{amsmath}
%% ============================== Useful packages ============================= %
\usepackage[dvipsnames,table,xcdraw]{xcolor}
\usepackage[round]{natbib}
\usepackage{amsmath}
\usepackage{graphicx}
\usepackage[colorlinks=true, allcolors=blue]{hyperref}
\usepackage{authblk}
\usepackage{float}
\usepackage{tikz}

%% Sets page size and margins
%\usepackage[a4paper, top=3cm,bottom=2cm,left=3cm,right=3cm,marginparwidth=4cm]{geometry}
\usepackage[papersize={10in, 12in}, top=3cm,bottom=2cm,left=2in,right=2in,marginparwidth=1.8in]{geometry}


\begin{document}



%% ----------------------------- information flow -----------------------------
informational closure


Information flow $J_{t}$ from environment to a system at time $t$ can be defined as
\begin{equation}\label{eq:InformationFlow}
\left.\begin{array}
{rl}{J_{t}(E \rightarrow Y )} & {:= I(Y_{t+1};E_{t}|Y_{t})} \\
{ } & { \ = H(Y_{t+1}|Y_{t})-H(Y_{t+1}|Y_{t},E_{t})} \\
{ } & { \ = H(E_{t}|Y_{t})-H(Y_{t}|Y_{t},Y_{t+1})} \\
{ } & { \ = H(E_{t}|Y_{t})-H(E_{t}|Y_{t},Y_{t+1})}\\
{ } & { \ = I(Y_{t+1};E_{t}) - (I(Y_{t+1};Y_{t})-I(Y_{t+1};Y_{t}|E_{t}))}
\end{array}\right.
\end{equation}



Rewrite 
\begin{equation}
I(Y_{t+1};E_{t}|Y_{t}) = 
I(Y_{t+1};E_{t}) - (I(Y_{t+1};Y_{t})-I(Y_{t+1};Y_{t}|E_{t}))
\end{equation}



% ---------------------------------- Trivial --------------------------------- %
Trivial case
\begin{equation}
\left.\begin{array}
{l}{I(Y_{t+1};E_{t})=0}\\
{I(Y_{t+1};Y_{t})-I(Y_{t+1};Y_{t}|E_{t})=0}
\end{array}\right.
\end{equation}


\begin{equation}
\begin{aligned}
{l}{I(Y_{t+1};E_{t})=0}&&{\Rightarrow}&&{J_{t}(E \rightarrow Y )=0}
\end{aligned}
\end{equation}




% -------------------------------- Define NTIC ------------------------------- %
\noindent [[ NTIC ]]
%\begin{equation}
%\left.\begin{array}
%{rl}{NTIC} & {:=I(Y_{t+1};E_{t},\dots,E_{t-m})-I(Y_{t+1};E_{t},\dots,E_{t-m}|Y_{t})}\\
%{ } & {=I(Y_{t+1};E_{t},\dots,E_{t-m})-I(Y_{t+1};E_{t}|Y_{t})}
%\end{array} \right.
%\end{equation}

\noindent In non-trivial cases, 
\begin{equation}
I(Y_{t+1};E_{t})\neq0, 
\end{equation}
This suggests that the process encodes information about the future state of the environment.\\
Therefore, informational closure can be achieved by

\begin{equation}
I(Y_{t+1};Y_{t})-I(Y_{t+1};Y_{t}|E_{t}) >0
\end{equation}

\noindent And non-trivial information closure (NTIC) can be defined as

\begin{equation}
\left.\begin{array}
{rl}{NTIC} & {:=I(Y_{t+1};Y_{t})-I(Y_{t+1};Y_{t}|E_{t})}\\
{ } & {\ =I(Y_{t+1};E_{t})-I(Y_{t+1};E_{t}|Y_{t})}
\end{array} \right.
\end{equation}


% --------------------------------- Maximise --------------------------------- %


%\begin{equation}
%max(NTIC) = max(I(Y_{t+1};Y_{t}))\ \&\ min(I(Y_{t+1};Y_{t}|E_{t}))
%\end{equation}
%
%
%\begin{equation}
%\begin{aligned}
%& maximise & NTIC & \\
%& s.t. & maximise & I(Y_{t+1};Y_{t})\\
%& & minimise & I(Y_{t+1};Y_{t}|E_{t})
%\end{aligned}
%\end{equation}




% --------------------------------- Maximise --------------------------------- %
%\begin{equation}
%\text{To maximize } NTIC \text{ is equivalent to}
%\end{equation}

\noindent To maximize $NTIC$ is equivalent to

\begin{equation}
\begin{aligned}
& \text{maximise} & { } & I(Y_{t+1};Y_{t}) & { } & and \\
& \text{minimise} & { } & I(Y_{t+1};Y_{t}|E_{t}) & { }
\end{aligned}
\end{equation}

%\begin{equation}
%\left.\begin{array}
%%{rl}{maximize} & {NTIC}\\
%%{maximize} & {I(Y_{t+1};Y_{t})}\\
%%{minimize} & {I(Y_{t+1};Y_{t}|E_{t})}
%
%
%
%\end{array} \right.
%\end{equation}


\cite{guttenberg2016neural}
\cite{BERTSCHINGER.2006}
\cite{jackendoff1987consciousness}
\cite{hoel2013quantifying}
\cite{hoel2016can}


\bibliographystyle{authordate1}
\bibliography{ref}



\end{document}